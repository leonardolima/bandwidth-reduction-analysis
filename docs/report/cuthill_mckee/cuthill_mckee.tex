<!-- wp:paragraph -->
<p>There are two possible formulations of the bandwidth minimization problem:</p>
<!-- /wp:paragraph -->

<!-- wp:heading {"level":3} -->
<h3>Matrix Bandwidth Minimization Problem</h3>
<!-- /wp:heading -->

<!-- wp:paragraph -->
<p>Let's first consider a matrix $A$, which is sparse and symmetric. The former means that most of its elements are equal to zero, while the latter means that $A$ is equal to its transpose. These assumptions are crystal clear for those whom have dealt with the problem before, but don't worry if this is not your case, the reason will be clear just a few paragraphs from here.</p>
<!-- /wp:paragraph -->

<!-- wp:paragraph -->
<p>Next, we can come up with the definition of the bandwidth of $A$:</p>
<!-- /wp:paragraph -->

<!-- wp:html -->
$$ B(A) = \max\{|i - j|: a_{ij} \neq 0 \} $$
<!-- /wp:html -->

<!-- wp:paragraph -->
<p>Using words, $B(A)$ is equal to the maximum distance of a nonzero element from the main diagonal of the matrix $A$. Our goal is to minimize this distance as much as possible, <em>i.e.</em>, we are trying to bring all nonzero elements of the matrix as close as possible to the main diagonal. </p>
<!-- /wp:paragraph -->

<!-- wp:paragraph -->
<p>In order to get a better grasp of the definition, consider the following matrix:</p>
<!-- /wp:paragraph -->

<!-- wp:html -->
$$
A = 
\begin{pmatrix}
 0 &amp; 1 &amp; 1 &amp; 0 &amp; 0\\ 
 1 &amp; 0 &amp; 0 &amp; 0 &amp; 0\\ 
 1 &amp; 0 &amp; 0 &amp; 0 &amp; 0\\
 0 &amp; 0 &amp; 0 &amp; 0 &amp; 1\\
 0 &amp; 0 &amp; 0 &amp; 1 &amp; 0\\
 \end{pmatrix}
$$
<!-- /wp:html -->

<!-- wp:paragraph -->
<p>$A$ is clearly sparse and symmetric, there is no doubt to that. Obviously, since $A$ is symmetric, we only need to consider the upper triangular part of the matrix when computing $B(A)$:</p>
<!-- /wp:paragraph -->

<!-- wp:html -->
$$ B(A) = \max\{|1 - 2|, |1 - 3|, |4 - 5|\} = 2 $$
<!-- /wp:html -->

<!-- wp:heading {"level":3} -->
<h3>Graph Bandwidth Minimization Problem</h3>
<!-- /wp:heading -->

<!-- wp:paragraph -->
<p>There exists a second formulation of the problem, which is related to graphs. Here we consider a graph $G$, specified by a set $V$ of vertices and a set $E$ of edges. The graph is assumed to be finite and undirected. We also assume an injective function $f: V \rightarrow {1,2,\dots,n}$, which goes from the set $V$ of vertices to the set of numbers from 1 up to $n$, where $n$ is the total number of vertices of the graph. This function $f$ is responsible for labeling the vertices of the graph, <em>i.e.</em>, for each vertex we assign a particular number from 1 up to $n$.</p>
<!-- /wp:paragraph -->

<!-- wp:paragraph -->
<p>The bandwidth of a particular vertex is defined as the maximum difference between its label and the label of its neighbors, or formally,</p>
<!-- /wp:paragraph -->

<!-- wp:html -->
$$ B_f(v) = \max_{i:(i,j) \in E }\{|f(i) - f(j)|\} $$
<!-- /wp:html -->

<!-- wp:paragraph -->
<p>Similarly, the bandwidth of the whole graph is defined as the maximum bandwidth among all vertices of the graph:</p>
<!-- /wp:paragraph -->

<!-- wp:html -->
$$ B_f(G) = \max_{v \in V} B_f(v) $$
<!-- /wp:html -->

<!-- wp:paragraph -->
<p>So, in order to compute the bandwidth of the graph first we compute the bandwidth of each vertex and then the maximum of all these computations is gonna be the bandwidth of the whole graph. In this case, we want to find a labeling $f$ in order to minimize the bandwidth of the graph.</p>
<!-- /wp:paragraph -->

<!-- wp:heading {"level":3} -->
<h3>Equivalence between formulations &amp; Complexity</h3>
<!-- /wp:heading -->

<!-- wp:paragraph -->
<p>The graph and matrix bandwidth minimization problems are equivalent. They are interconvertible, so it does not matter which formulation we decide to work with.</p>
<!-- /wp:paragraph -->

<!-- wp:paragraph -->
<p>This is a long-established combinatorial optimization problem that has been around for more than 60 years. Unfortunately, it is NP-complete, which means that it can't be solved in polynomial time in any known way.</p>
<!-- /wp:paragraph -->

<!-- wp:paragraph -->
<p>If we consider a graph with $n$ vertices, the number of possible labeling is $n!$, thus trying all possible permutations has running time complexity of $\mathcal{O}(n!)$. Even for a small graph with only $10$ vertices this approach is not practical. Therefore, we need to find an approach which not only gives us a good approximation of the optimal solution, it also has to be feasible.</p>
<!-- /wp:paragraph -->

<!-- wp:heading {"level":3} -->
<h3>Cuthill–McKee algorithm</h3>
<!-- /wp:heading -->

<!-- wp:paragraph -->
<p>One very simple approach is called the Cuthill–McKee algorithm <strong>[1]</strong>, which is better explained alongside with an example. Consider, for instance, the graph below:</p>
<!-- /wp:paragraph -->

<!-- wp:image {"align":"center","id":71,"width":410,"height":222,"sizeSlug":"large"} -->
<div class="wp-block-image"><figure class="aligncenter size-large is-resized"><img src="http://www.leonardolima.net/wp-content/uploads/2019/11/graph1-1024x553.png" alt="" class="wp-image-71" width="410" height="222"/></figure></div>
<!-- /wp:image -->

<!-- wp:paragraph -->
<p>Assuming that $\{A, B, \ldots, H\} \to \{1, 2, \ldots, 10\}$ where $\{1, 2, \ldots, 10\}$ are the indices of rows and columns, the graph has the following adjacency matrix:</p>
<!-- /wp:paragraph -->

<!-- wp:html -->
$$
M = 
\begin{pmatrix}
 1 &amp; 1 &amp; 0 &amp; 1 &amp; 0 &amp; 0 &amp; 0 &amp; 0 &amp; 1 &amp; 0\\ 
 1 &amp; 1 &amp; 1 &amp; 0 &amp; 0 &amp; 0 &amp; 0 &amp; 0 &amp; 1 &amp; 0\\ 
 0 &amp; 1 &amp; 1 &amp; 0 &amp; 1 &amp; 0 &amp; 0 &amp; 0 &amp; 1 &amp; 0\\
 1 &amp; 0 &amp; 0 &amp; 1 &amp; 1 &amp; 1 &amp; 0 &amp; 0 &amp; 1 &amp; 1\\ 
 0 &amp; 0 &amp; 1 &amp; 1 &amp; 1 &amp; 0 &amp; 0 &amp; 1 &amp; 1 &amp; 1\\ 
 0 &amp; 0 &amp; 0 &amp; 1 &amp; 0 &amp; 1 &amp; 1 &amp; 0 &amp; 0 &amp; 1\\ 
 0 &amp; 0 &amp; 0 &amp; 0 &amp; 0 &amp; 1 &amp; 1 &amp; 1 &amp; 0 &amp; 1\\ 
 0 &amp; 0 &amp; 0 &amp; 0 &amp; 1 &amp; 0 &amp; 1 &amp; 1 &amp; 0 &amp; 1\\ 
 1 &amp; 1 &amp; 1 &amp; 1 &amp; 1 &amp; 0 &amp; 0 &amp; 0 &amp; 1 &amp; 1\\ 
 0 &amp; 0 &amp; 0 &amp; 1 &amp; 1 &amp; 1 &amp; 1 &amp; 1 &amp; 1 &amp; 1\\ 
 \end{pmatrix}
$$
<!-- /wp:html -->

<!-- wp:paragraph -->
<p>The bandwidth of $M$ in this particular case is equal to $8$.</p>
<!-- /wp:paragraph -->

<!-- wp:paragraph -->
<p>Before starting the procedure, we should first define the degree of a vertex $i$, $deg(i)$, which is the number of edges meeting at $i$. Considering a matrix, it is the number of non-zero elements outside of the main diagonal.</p>
<!-- /wp:paragraph -->

<!-- wp:paragraph -->
<p>Let's start the procedure by choosing a starting vertex. In general, it is a good choice to choose the one with lowest degree (or one of them if there is more), but sometimes this does not lead to the optimal result. $deg(A) = 3$ and we can check that $3$ is in fact the lowest degree here, so we can start labeling $A$ as $1$, where we use the notation $A_1$.</p>
<!-- /wp:paragraph -->

<!-- wp:paragraph -->
<p>After choosing the first vertex, we check if its neighbors are already labeled. If not, we label them accordingly, from lowest to highest degree. In this case, the neighbors are ordered as $B, D, I$, because $deg(B) = 3$, $deg(D) = 5$ and $deg(I) = 6$. At this point, our procedure gives us the following graph:</p>
<!-- /wp:paragraph -->

<!-- wp:image {"align":"center","id":78,"width":230,"height":136,"sizeSlug":"large"} -->
<div class="wp-block-image"><figure class="aligncenter size-large is-resized"><img src="http://www.leonardolima.net/wp-content/uploads/2019/11/graph2.png" alt="" class="wp-image-78" width="230" height="136"/></figure></div>
<!-- /wp:image -->

<!-- wp:paragraph -->
<p>Next, we do the exact same procedure all over again considering the labeling order, <em>i.e.</em>, we start by checking the neighbors of $B$. $A$ and $I$ are already labeled, but $C$ is not, thus we label it as $5$. We execute the exact same procedure for the neighbors of $D$, which are $A, I, E, J$ and $F$, but only $E, J$ and $F$ are not already labeled. $deg(E) = 5$, $deg(J) = 6$ and $deg(F) = 3$, so we label them accordingly and get the following graph:</p>
<!-- /wp:paragraph -->

<!-- wp:image {"align":"center","id":79,"width":410,"height":226,"sizeSlug":"large"} -->
<div class="wp-block-image"><figure class="aligncenter size-large is-resized"><img src="http://www.leonardolima.net/wp-content/uploads/2019/11/graph3-1024x563.png" alt="" class="wp-image-79" width="410" height="226"/></figure></div>
<!-- /wp:image -->

<!-- wp:paragraph -->
<p>Again, we repeat the same procedure for vertex $C$, but its neighbors are already labeled, thus we go to the next vertex. Considering $F$, from the set of its neighbors, only $G$ doesn't have a label, so we label it as $9$. Finally, we do the same thing for $J$ and label $H$ as $10$, leaving no vertices unlabeled, and giving us the following graph as a result:</p>
<!-- /wp:paragraph -->

<!-- wp:image {"align":"center","id":80,"width":410,"height":227,"sizeSlug":"large"} -->
<div class="wp-block-image"><figure class="aligncenter size-large is-resized"><img src="http://www.leonardolima.net/wp-content/uploads/2019/11/graph_final-1024x565.png" alt="" class="wp-image-80" width="410" height="227"/></figure></div>
<!-- /wp:image -->

<!-- wp:paragraph -->
<p>If now we construct the adjacency matrix considering not the alphabetic order of the letters as rows and columns, but the labels, in increasing order, we get the following adjacency matrix:</p>
<!-- /wp:paragraph -->

<!-- wp:html -->
$$
M = 
\begin{pmatrix}
 1 &amp; 1 &amp; 1 &amp; 1 &amp; 0 &amp; 0 &amp; 0 &amp; 0 &amp; 0 &amp; 0\\ 
 1 &amp; 1 &amp; 0 &amp; 1 &amp; 1 &amp; 0 &amp; 0 &amp; 0 &amp; 0 &amp; 0\\ 
 1 &amp; 0 &amp; 1 &amp; 1 &amp; 0 &amp; 1 &amp; 1 &amp; 1 &amp; 0 &amp; 0\\
 1 &amp; 1 &amp; 1 &amp; 1 &amp; 1 &amp; 0 &amp; 1 &amp; 1 &amp; 0 &amp; 0\\ 
 0 &amp; 1 &amp; 0 &amp; 1 &amp; 1 &amp; 0 &amp; 1 &amp; 0 &amp; 0 &amp; 0\\ 
 0 &amp; 0 &amp; 1 &amp; 0 &amp; 0 &amp; 1 &amp; 0 &amp; 1 &amp; 1 &amp; 0\\ 
 0 &amp; 0 &amp; 1 &amp; 1 &amp; 1 &amp; 0 &amp; 1 &amp; 1 &amp; 0 &amp; 1\\ 
 0 &amp; 0 &amp; 1 &amp; 1 &amp; 0 &amp; 1 &amp; 1 &amp; 1 &amp; 1 &amp; 1\\ 
 0 &amp; 0 &amp; 0 &amp; 0 &amp; 0 &amp; 1 &amp; 0 &amp; 1 &amp; 1 &amp; 1\\ 
 0 &amp; 0 &amp; 0 &amp; 0 &amp; 0 &amp; 0 &amp; 1 &amp; 1 &amp; 1 &amp; 1\\ 
 \end{pmatrix}
$$
<!-- /wp:html -->

<!-- wp:paragraph -->
<p>Hence, after the procedure, the bandwidth of the matrix was reduced to $5$. One can easily visualize that the non-zero elements are in fact closer to the main diagonal.</p>
<!-- /wp:paragraph -->

<!-- wp:paragraph -->
<p>The above procedure was implemented by me in C++. It can be found <a href="https://github.com/leonardolima/peak-memory-optimization/blob/master/src/bandwidth_minimization.cpp">here</a>.</p>
<!-- /wp:paragraph -->

<!-- wp:paragraph -->
<p><strong>[1]</strong> Cuthill, E. and McKee, J., 1969, August. Reducing the bandwidth of sparse symmetric matrices. In&nbsp;<em>Proceedings of the 1969 24th national conference</em>&nbsp;(pp. 157-172). ACM.</p>
<!-- /wp:paragraph -->
